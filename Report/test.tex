\documentclass[a4paper,12pt]{report}
\usepackage[margin=1in]{geometry} % 1-inch margins
\usepackage{tikz} % For the border
\usetikzlibrary{calc} % For coordinate calculations
\usepackage{graphicx} % For including the logo
\usepackage{lmodern} % Font package
\usepackage{titlesec} % Section formatting
\usepackage{fancyhdr} % Footer customization
\usepackage{amsmath}
\usepackage{listings}
\usepackage{caption}
\usepackage{subcaption}
\usepackage{geometry}
\usepackage{array}
\usepackage{booktabs}
\usepackage{makecell}
\usepackage{float}
\usepackage{tocloft}
\usepackage[colorlinks=true, linkcolor=black, citecolor=black, urlcolor=black]{hyperref}
\usepackage{lipsum}
\usepackage{adjustbox} % trong phần preamble
\usepackage{multicol}
\usepackage{indentfirst}

\pagestyle{fancy}
\lhead{S\&P 500 Crisis Analysis}
\definecolor{codegreen}{rgb}{0,0.6,0}
\definecolor{codegray}{rgb}{0.5,0.5,0.5}
\definecolor{codepurple}{rgb}{0.58,0,0.82}
\definecolor{backcolour}{rgb}{0.95,0.95,0.92}

\lstdefinestyle{mystyle}{
    backgroundcolor=\color{backcolour},   
    commentstyle=\color{codegreen},
    keywordstyle=\color{magenta},
    numberstyle=\tiny\color{codegray},
    stringstyle=\color{codepurple},
    basicstyle=\ttfamily\footnotesize,
    breakatwhitespace=false,         
    breaklines=true,                 
    captionpos=b,                    
    keepspaces=true,                 
    numbers=left,                    
    numbersep=5pt,  
    showlines=false,
    showspaces=false,                
    showstringspaces=false,
    showtabs=false,                  
    tabsize=2
}

\lstset{ % Tùy chỉnh các thuộc tính
    basicstyle=\ttfamily\small, % Kiểu chữ máy tính nhỏ
    backgroundcolor=\color{lightgray}, % Màu nền nhẹ cho code
    keywordstyle=\color{blue}\bfseries, % Màu xanh dương cho từ khóa, in đậm
    commentstyle=\color{green}, % Màu xanh lá cho comment
    stringstyle=\color{purple}, % Màu tím cho chuỗi
    numberstyle=\tiny\color{gray}, % Màu xám cho số dòng
    numbers=left, % Hiển thị số dòng bên trái
    stepnumber=1, % Đánh số mỗi dòng
    numbersep=5pt, % Khoảng cách giữa số dòng và mã
    showstringspaces=false, % Không hiển thị dấu cách trong chuỗi
    breaklines=true, % Ngắt dòng tự động
    captionpos=b, % Đặt caption ở dưới code
    frame=single, % Đặt khung xung quanh code
}


\begin{document}
\thispagestyle{empty}  % Xóa số trang của trang đầu tiên

% --- First Page Border ---
\begin{tikzpicture}[remember picture, overlay]
    \draw[line width=2pt]
        ($(current page.south west) + (0.5in, 0.5in)$) 
        rectangle 
        ($(current page.north east) - (0.5in, 0.5in)$);
\end{tikzpicture}

% --- University Name ---
\begin{center}
    
    {\Large \textbf{University of Science and Technology of Hanoi}}\\[0.75cm]
    
    % --- University Logo ---
    
\begin{center}
  \makebox[\textwidth][c]{%
    \hspace*{1.2cm} % dịch logo sang trái 1cm, bạn thử chỉnh -0.8cm hoặc -1.2cm nếu cần
    \includegraphics[width=0.7\textwidth]{Img/usth.png}
  }
\end{center}

    
    \vspace{1cm}
    {\LARGE \textbf{Fundamentals of Data Science}}\\[0.75cm]

    % --- Assignment Title ---
    \vspace{0.5cm} % khoảng cách trước Labwork I
    {\huge \textbf{S\&P 500 Crisis Analysis}}
    \vspace{1.5cm} % khoảng cách sau Labwork I
\end{center}

% --- Student Information ---
\vspace{0.5cm}
\noindent
\begin{center}

    \begin{tabular}{l c r}
    
        \textbf{\large Group Members:}
        \vspace{0.5cm}
        \\
        \textbf{Nguyen Tuan Thanh} & 22BA13289 & thanhnt.22ba13289@usth.edu.vn
        \vspace{0.1cm}\\
        \textbf{Nguyen Chi Quang} & 22BA13262 & quangnc.22ba13262@usth.edu.vn \\
    \end{tabular}
\end{center}

\vfill

% --- Footer ---
\begin{center}
    \textit{Hanoi, Sep 2025}
\end{center}
\newpage
\pagenumbering{arabic} 
\setcounter{page}{2}  

% Đặt lại số trang về 1 cho mục lục và các phần sau

\tableofcontents  % Tạo mục lục tự động

\newpage

% ---------------- DÀN Ý ----------------
\chapter*{ACKNOWLEDGEMENTS}
\addcontentsline{toc}{chapter}{ACKNOWLEDGEMENTS}

\chapter*{ABSTRACT}
\addcontentsline{toc}{chapter}{ABSTRACT}

\chapter{Introduction}

\chapter{Business Analysis}
\section{Background}
\section{Problem Statement}
\section{Analytical Approach}
\chapter{Data Collection and Management}
\section{Introduction}
The dataset employed in this project is the Standard\& Poor’s (S\&P) 500 Index data. It contains monthly historical records with key financial indicators such as Dividend, Earnings, and Price-to-Earnings (P/E) Ratio. The dataset consists of 1833 entries and 10 features, and is provided in CSV (Comma Separated Values) format.
\begin{table}[H]
\centering
\begin{adjustbox}{width=\textwidth}
\begin{tabular}{|c|c|c|c|c|c|c|c|c|c|}
\hline
\textbf{Date} & \textbf{SP500} & \textbf{Dividend} & \textbf{Earnings} & \textbf{CPI} & \textbf{LIR} & \textbf{Real Price} & \textbf{Real Dividend} & \textbf{Real Earnings} & \textbf{PE10}\\
\hline
2023-06-01 & 4345.37 & 68.71 & 181.17 & 305.11 & 3.75 & 4359.88 & 68.94 & 181.77 & 32.41 \\
2023-05-01 & 4146.17 & 68.54 & 179.17 & 304.13 & 3.57 & 4173.45 & 68.99 & 180.35 & 31.14 \\
2023-04-01 & 4121.47 & 68.38 & 177.17 & 303.36 & 3.46 & 4159.03 & 69.00 & 178.78 & 31.15 \\
...&...&...&...&...&...&...&...&...&...\\
2017-09-01 & 2492.84 & 48.17 & 107.08 & 246.82 & 2.20 & 3091.85 & 59.74 & 132.81 & 32.97 \\
2017-08-01 & 2456.22 & 47.85 & 106.06 & 245.52 & 2.21 & 3062.56 & 59.67 & 132.24 & 32.71 \\
\hline
\end{tabular}
\end{adjustbox}
\caption{S\&P 500 Dataset}
\end{table}
The data types and detailed discussion of each feature in the dataset are provided below:


\begin{table}[H]
\centering
\begin{adjustbox}{width=\textwidth}
\begin{tabular}{|c|c|c|l|}
\hline
\textbf{Features} & \textbf{Type} & \textbf{Data Type} & \textbf{Explanation}\\
\hline Date & datetime & Qualitative  & Timestamp of the record \\ \hline
\hline SP500  & float & Qualitative  & The nominal value at that time. \\ \hline
\hline Dividend & float & Qualitative  & The average dividend payout of companies. \\ \hline
\hline Earnings & float & Qualitative  & The average earnings of companies. \\ \hline
\hline CPI & float & Qualitative  & A measure of inflation and purchasing power. \\ \hline
\hline LIR & float & Qualitative  & The 10-year government bond yield. \\ \hline
\hline Real Price & float & Qualitative  & The inflation-adjusted S\&P 500 index value (using CPI). \\ \hline
\hline Real Dividend & float & Qualitative  & Dividend values adjusted for inflation. \\ \hline
\hline Real Earnings & float & Qualitative  & Earnings adjusted for inflation. \\ \hline
\hline PE10 & float & Qualitative  & The cyclically adjusted price-to-earnings ratio. \\ \hline
\end{tabular}
\end{adjustbox}
\caption{Features Explanation}
\end{table}
\section{Problem Context}
\section{Overall Architecture}
\section{Data Collection}
\section{Data Management}
\section{Data Processing}
\chapter{Statistics}
\section{Descriptive Statistics}
\subsection{Full Sample}
\subsection{Period Analysis}
\section{Distribution Analysis}
\section{Correlation}
\section{Hypothesis Testing}
\chapter{Visualization}
\section{Market Performance}
\section{Valuation Metrics}
\section{Risk}
\section{Dividends}
\chapter*{Conclusion and Future Work}
\chapter{References}
\end{document}